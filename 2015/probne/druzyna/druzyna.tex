\documentclass{article}

\usepackage{amssymb}
\usepackage{polski}
\usepackage[polish]{babel}
\usepackage[utf8]{inputenc}
\usepackage[T1]{fontenc}
\usepackage{libertine}
\usepackage{graphicx}
\usepackage{multicol}
\usepackage{fancyhdr}
\usepackage[a4paper,headheight=65pt,bottom=65pt]{geometry}
\usepackage{lastpage}

\pagestyle{fancy}
\fancyhf{}
\fancyhead[L]{Podlaski Turniej w~Programowaniu Zespołowym 2015\\ Politechnika Białostocka, 21-22 marca 2015}
\fancyhead[R]{\includegraphics[scale=0.7]{logoturnieju.png}}
\fancyfoot[L]{Problem DRU: Drużyna}
\fancyfoot[R]{\thepage/\pageref{LastPage}}
\renewcommand{\footrulewidth}{0.4pt}

\begin{document}
\begin{center}
  \begin{Huge}
    Problem DRU: Drużyna
  \end{Huge}
\end{center}

W~pewnym egzotycznym kraju postanowiono stworzyć drużynę koszykówki. Z~racji tego, że
rozrywek tam mają niewiele, to zgłosiło się bardzo wielu chętnych. Sam trener też nie ma wielkiego
pojęcia o~tym sporcie i~w~ramach rekrutacji kieruje się tylko jednym kryterium – wzrostem.
Zgodnie z~jego pomysłem na parkiecie zawsze ma grac pięciu najwyższych zawodników. Jest tak
wielu chętnych, że trener nie poradzi sobie z~tym zadaniem bez pomocy komputera i~oczywiście
dobrego programisty.

\section*{Wejście}

W~pierwszej linii wejścia podana jest liczba całkowita n ($5 \leqslant n \leqslant 10^5$) oznaczająca liczbę
kandydatów do drużyny. W~każdej z~kolejnych n linii pojawi się jedna liczba całkowita w
($100 \leqslant w \leqslant 250$) oznaczająca wzrost danego kandydata.

\section*{Wyjście}

W~jedynej linii wyjścia ma się pojawić jedna liczba całkowita – oznaczająca średni wzrost
zawodników wybranych do gry. Można zakładać, że średni wzrost zawsze będzie liczbą całkowitą.

\section*{Przykład}
dane wejściowe:
\begin{verbatim}
10
182
151
212
177
224
199
204
191
165
143
\end{verbatim}
wynik:
\begin{verbatim}
206
\end{verbatim}
\end{document}
