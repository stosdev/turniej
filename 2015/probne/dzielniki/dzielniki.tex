\documentclass{article}

\usepackage{amssymb}
\usepackage{polski}
\usepackage[polish]{babel}
\usepackage[utf8]{inputenc}
\usepackage[T1]{fontenc}
\usepackage{libertine}
\usepackage{graphicx}
\usepackage{multicol}
\usepackage{fancyhdr}
\usepackage[a4paper,headheight=65pt,bottom=65pt]{geometry}
\usepackage{lastpage}

\pagestyle{fancy}
\fancyhf{}
\fancyhead[L]{Podlaski Turniej w~Programowaniu Zespołowym 2015\\ Politechnika Białostocka, 21-22 marca 2015}
\fancyhead[R]{\includegraphics[scale=0.7]{logoturnieju.png}}
\fancyfoot[L]{Problem DZI: Dzielniki}
\fancyfoot[R]{\thepage/\pageref{LastPage}}
\renewcommand{\footrulewidth}{0.4pt}

\begin{document}
\begin{center}
  \begin{Huge}
    Problem DZI: Dzielniki
  \end{Huge}
\end{center}

Pani z~matematyki ma problemy z~utrzymaniem spokoju na szkolnych zajęciach. W~końcu
zdecydowała, że dzieci muszą zająć się jakimś ciekawym dla nich tematem. Dzielniki liczby
okazały się zagadnieniem bardzo absorbującym uwagę dzieci. Pani podaje przedział liczb, a~dzieci
mają odpowiedzieć na pytanie, która liczba z~tego przedziału ma najwięcej dzielników. Zwycięzca
dostaje piątkę i~cukierka. Mały Jaś jest sprytny i~chce znaleźć osobę, która napisze mu aplikację na
telefon komórkowy, co pozwoli mu zawsze wygrywać. Zwraca się z~prośbą do Ciebie.

\section*{Wejście}

W~pierwszej linii wejścia podana jest liczba całkowita n ($1 \leqslant n \leqslant 10$) oznaczająca liczbę zapytań
nauczycielki. W~kolejnych n liniach znajdują się po dwie liczby a~i~b ($1 \leqslant a \leqslant b \leqslant 100$) oznaczające
dolny i~górny zakres szukanego przedziału.

\section*{Wyjście}

W~każdej z~n linii wyjścia ma się pojawić odpowiedni wynik: liczba posiadająca najwięcej
dzielników z~wszystkich liczb należących do przedziału <a, b>. Jeżeli istnieje więcej niż jedna
liczba z~największą ilością dzielników, to trzeba wypisać najmniejszą z~nich.i

\section*{Przykład}
dane wejściowe:
\begin{verbatim}
3
1 50
60 80
5 15
\end{verbatim}
wynik:
\begin{verbatim}
48
60
12
\end{verbatim}
\end{document}
