\documentclass{article}

\usepackage{amssymb}
\usepackage{polski}
\usepackage[polish]{babel}
\usepackage[utf8]{inputenc}
\usepackage[T1]{fontenc}
\usepackage{libertine}
\usepackage{graphicx}
\usepackage{multicol}
\usepackage{fancyhdr}
\usepackage[a4paper,headheight=65pt,bottom=65pt]{geometry}
\usepackage{lastpage}

\pagestyle{fancy}
\fancyhf{}
\fancyhead[L]{Podlaski Turniej w~Programowaniu Zespołowym 2015\\ Politechnika Białostocka, 21-22 marca 2015}
\fancyhead[R]{\includegraphics[scale=0.7]{logoturnieju.png}}
\fancyfoot[L]{Problem E: Kasa misiu, kasa}
\fancyfoot[R]{\thepage/\pageref{LastPage}}
\renewcommand{\footrulewidth}{0.4pt}

\begin{document}
\begin{center}
  \begin{Huge}
    Problem E: Kasa misiu, kasa
  \end{Huge}
\end{center}

W~Polsce XXII wieku płaci się w~zasadzie za wszystko. W~celu załatania dziury budżetowej
wprowadzono nawet podatek od oddychania i~od mówienia. Dużo płaci się również za posiadanie
dzieci oraz za brak dzieci. Nie inaczej sprawa się ma z~drogami -- przejazd każdą z~nich (nieważne
czy to autostrada, czy polny dukt) jest obciążony pewną opłatą, którą należy uiścić w~specjalnej
budce przy wjeździe na dany odcinek drogi. Stąd kierowców (oczywiście tych, którzy jeszcze nie
dali nogi z~kraju) interesuje nie tylko jak najkrótszy czas podróży, ale również jej koszt. W
internecie powstaje nawet specjalny portal, którego zadaniem jest wyszukiwanie najbardziej
korzystnych połączeń, które nie zrujnują budżetu przeciętnego Kowalskiego. Twórcy portalu
zgłosili się do Ciebie z~prośbą o~napisanie odpowiedniego algorytmu. Celem jest znalezienie
najszybszego przejazdu pomiędzy zadaną parą miast, którego łączny koszt nie przekracza zadanego
budżetu.

\section*{Wejście}

W~pierwszej linijce wejścia są podane są trzy liczby całkowite $n$, $m$, $b$ ($2\leqslant n \leqslant 3000$,
$1\leqslant m \leqslant 30000$, $1\leqslant b \leqslant 3000$) oznaczające odpowiednio: liczbę miast, liczbę łączących je
dwukierunkowych dróg oraz dostępny budżet. W~kolejnych $m$ liniach pojawią się informacje o
kolejnych drogach: w~każdej z~linii po cztery liczby całkowite $x$, $y$, $t$, $c$ ($1\leqslant x$, $y\leqslant n$, $1\leqslant t\leqslant 1000$,
$1\leqslant c\leqslant 1000$) oznaczające odpowiednio numery połączonych miast, czas przejazdu oraz koszt
przejazdu. W~ostatniej linii wejścia podane są dwie liczby całkowite $s$, $e$ ($1\leqslant s$, $e\leqslant n$) oznaczające
miasto początkowe i~miasto końcowe podróżny.

\section*{Wyjście}
W jedynej linii wejścia ma się pojawić czas najkrótszej podróży, której koszt nie przekracza
budżetu.

Można założyć, że rozwiązanie zawsze istnieje.

\section*{Przykład}
dane wejściowe:
\begin{verbatim}
8 15 7
1 2 3 2
1 3 5 1
1 4 4 1
2 3 1 3
2 5 3 3
2 7 4 1
3 4 2 3
3 7 2 1
4 6 5 2
4 7 5 1
5 7 8 1
5 8 3 3
6 7 3 1
6 8 4 2
7 8 7 1
1 8
\end{verbatim}
wynik:
\begin{verbatim}
13
\end{verbatim}
\end{document}
