\documentclass{article}

\usepackage{amssymb}
\usepackage{polski}
\usepackage[polish]{babel}
\usepackage[utf8]{inputenc}
\usepackage[T1]{fontenc}
\usepackage{libertine}
\usepackage{graphicx}
\usepackage{multicol}
\usepackage{fancyhdr}
\usepackage[a4paper,headheight=65pt,bottom=65pt]{geometry}
\usepackage{lastpage}

\pagestyle{fancy}
\fancyhf{}
\fancyhead[L]{Podlaski Turniej w~Programowaniu Zespołowym 2015\\ Politechnika Białostocka, 21-22 marca 2015}
\fancyhead[R]{\includegraphics[scale=0.7]{logoturnieju.png}}
\fancyfoot[L]{Problem S: Prezes Bajt}
\fancyfoot[R]{\thepage/\pageref{LastPage}}
\renewcommand{\footrulewidth}{0.4pt}

\begin{document}
\begin{center}
  \begin{Huge}
    Problem D: Prezes Bajt
  \end{Huge}
\end{center}

Prezes firmy XY pan Bajt dostał nagrodę ,,LIDERA BIZNESU'' za niekwestionowane osiągnięcia finansowe swojej firmy.
Tradycyjnie, na gali wręczenia nagrody, LIDER przedstawia krótką prezentację strategii prowadzenia firmy przed kolegami z
branży. Pan Bajt pracuje na stanowisku już od n miesięcy, więc zlecił przygotowanie szczegółowego zestawienia finansowego
opisującego ile firma zarobiła / straciła w~poszczególnych miesiącach. Pan Bajt zdecydował, że opowie jedynie o~tym okresie,
w~którym zarobił najwięcej. Poprosił informatyków o~wyselekcjonowanie takiego okresu. Szukany okres czasu musi być
nieprzerwany (tzn. pewna liczba bezpośrednio następujących po sobie miesięcy) oraz ,,najważniejsze'' -- suma zysków
uzyskanych podczas wyselekcjonowanych miesięcy musi być największa. Długość okresu nie ma znaczenia, ponieważ pan
Bajt prowadzi firmę niemal od początku a~jego sukces jest wyjątkowy.

\section*{Prosty przykład}

\begin{center}
\begin{tabular}{|c|c|c|c|c|c|c|c|c|}
  \hline
  6 & -10 & 5 & -1 & 4 & 8 & -3 & 9 & 11\\
  \hline
\end{tabular}
\end{center}

Największą sumę uzyskamy dodając zyski / straty z~miesięcy $3$, $4$, $5$, $6$, $7$, $8$ i~$9$, tj. $5 + (-1) + 4 + 8 + (-3) + 9 + 11 = 33$

\section*{Wejście}

W~pierwszym wierszu wejścia znajduje się całkowita liczba $n$ ($2 < n < 10^6$) określająca ilość miesięcy przepracowanych przez pana
Bajta na stanowisku prezesa.

W~kolejnych $n$ liniach umieszczone są liczby całkowite z~przedziału [$-10000, 10000$] reprezentujące straty / zyski firmy w
kolejnych $n$ miesiącach. Umawiamy się, że kolejne miesiące są numerowane od $1$ do $n$.

\section*{Wyjście}

W~pierwszej linii wyjścia zapisz pierwszy i~ostatni miesiąc okresu, w~którym pan Bajt uzyskał największy sukces (tj.
zarobił najwięcej pieniędzy) oraz sumę zysków / strat z~tego okresu. Wartości oddziel pojedynczą spacją.

\section*{Przykład}
\begin{multicols*}{2}
dane wejściowe:
\begin{verbatim}
15
51
10
-680
1
-68
10
3445
45
-10
1000
400
90
5645
546
564
\end{verbatim}
wynik:
\begin{verbatim}
6 15 11735
\end{verbatim}
\end{multicols*}
\end{document}
