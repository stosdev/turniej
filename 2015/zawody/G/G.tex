\documentclass{article}

\usepackage{amssymb}
\usepackage{polski}
\usepackage[polish]{babel}
\usepackage[utf8]{inputenc}
\usepackage[T1]{fontenc}
\usepackage{libertine}
\usepackage{graphicx}
\usepackage{multicol}
\usepackage{fancyhdr}
\usepackage[a4paper,headheight=65pt,bottom=65pt]{geometry}
\usepackage{lastpage}

\pagestyle{fancy}
\fancyhf{}
\fancyhead[L]{Podlaski Turniej w~Programowaniu Zespołowym 2015\\ Politechnika Białostocka, 21-22 marca 2015}
\fancyhead[R]{\includegraphics[scale=0.7]{logoturnieju.png}}
\fancyfoot[L]{Problem G: Wyścigi}
\fancyfoot[R]{\thepage/\pageref{LastPage}}
\renewcommand{\footrulewidth}{0.4pt}

\begin{document}
\begin{center}
  \begin{Huge}
    Problem G: Wyścigi
  \end{Huge}
\end{center}

Kazik i~jego koledzy są cool. Ich życie to wspaniałe samochody, wielka prędkość, adrenalina,
mocne wrażenia\ldots300 km/h na autostradzie pod prąd? Nic nowego. Pokonywanie zakrętów na
granicy śmierci? Chleb powszedni. Nielegalne wyścigi samochodowe to wielka frajda i~okazja do
poznania wielu ciekawych ludzi -- Kazik czasem nawet ściga się z~kierowcami z~innych części
świata. Można by pomyśleć, że szaleją za nim tabuny dziewczyn. Jest jednak inaczej, bo Kazio jest
przyklejony do monitora dniami i~nocami grając w~wyścigówki. Nie śpi, nie je i~nie pije. Tak
właśnie wygląda jego życie i~jego świat. Na całym świecie jest wielu takich jak on. Dzięki nim
pracownicy firm programistycznych, tacy jak Ty, zarabiają niezłe pieniądze. Wyniki
poszczególnych graczy umieszczane są na serwerze. Na podstawie zbiorczych danych należy
przygotować zestawienie dotyczące rekordów poszczególnych tras. Twoim zadaniem jest napisanie
programu, który dla każdej trasy określi, ile razy był ustanawiany na niej nowy rekord.

\section*{Wejście}

W~pierwszej linii wejścia podane są dwie liczby $n$ i~$m$ ($1\leqslant n\leqslant 10^6$, $1\leqslant m\leqslant 10$) oznaczające liczbę
przejazdów i~liczbę tras. W~każdej z~kolejnych $n$ linii podane są opisy poszczególnych przejazdów:
po trzy liczby całkowite $a$, $b$, $c$ ($1\leqslant a\leqslant m$, $1\leqslant b\leqslant 10^6$, $1\leqslant c\leqslant 10^6$) oznaczające odpowiednio
numer trasy, czas przejazdu oraz moment, w~którym ukończono trasę. Zakładamy, że na danej trasie
nie istnieją dwa przejazdy, które skończyły się w~tym samym momencie.

\section*{Wyjście}

W~$m$ liniach wyjścia powinny pojawić się informacje dotyczące liczby pobitych rekordów dla
kolejnych tras. W~wyliczeniach nie należy uwzględniać wyrównań rekordów.

\section*{Przykład}
dane wejściowe:
\begin{verbatim}
10 3
1 1008 4095
2 349 5900
1 911 3495
3 3945 394
2 407 3335
1 883 5239
1 883 6678
3 4023 500
2 313 7104
3 4499 989
\end{verbatim}
wynik:
\begin{verbatim}
2
3
1
\end{verbatim}
\end{document}
