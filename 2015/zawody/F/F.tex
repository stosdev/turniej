\documentclass{article}

\usepackage{amssymb}
\usepackage{fancyvrb}
\usepackage{polski}
\usepackage[polish]{babel}
\usepackage[utf8]{inputenc}
\usepackage[T1]{fontenc}
\usepackage{libertine}
\usepackage{graphicx}
\usepackage{multicol}
\usepackage{fancyhdr}
\usepackage[a4paper,headheight=65pt,bottom=65pt]{geometry}
\usepackage{lastpage}

\pagestyle{fancy}
\fancyhf{}
\fancyhead[L]{Podlaski Turniej w~Programowaniu Zespołowym 2015\\ Politechnika Białostocka, 21-22 marca 2015}
\fancyhead[R]{\includegraphics[scale=0.7]{logoturnieju.png}}
\fancyfoot[L]{Problem F: Modyfikacja koloru kwiatu tulibajtu}
\fancyfoot[R]{\thepage/\pageref{LastPage}}
\renewcommand{\footrulewidth}{0.4pt}

\begin{document}
\begin{center}
  \begin{Huge}
    Problem F: Modyfikacja koloru kwiatu tulibajtu
  \end{Huge}
\end{center}

W~instytucie genetyki roślin w~Bajtlandzie prowadzone są badania nad nowymi odmianami tulibajtu z~wykorzystaniem technik
inżynierii genetycznej. Botanicy zmierzają w~kierunku pozyskania wielu interesujących barw tego pięknego kwiatu. Wśród
setek barwników roślinnych wyróżnia się trzy główne grupy związków chemicznych. Są to karotenoidy, betalainy oraz
flawonoidy. Ostatnia z~tych grup ma największy wpływ na ostateczny kolor kwiatów. Do flawonoidów należy wiele klas
związków, takich jak antocyjaniny (odpowiedzialne za kolor pomarańczowy, czerwony, purpurowy, fioletowy i~niebieski),
aurony, chalkony (żółty), flawony i~flawonole (postrzegane jako bezbarwne lub bladożółte). Spośród znanych barwników
roślinnych wybrano podstawowe i~oznaczono je kolejnymi znakami alfabetu systemu 16-kowego, tj. 0, 1, \ldots, 9, A, B, \ldots, F.
Ostatnio prowadzone badania skupiają się na odpowiednim klasyfikowaniu i~krzyżowaniu materiału genetycznego różnych
odmian tulibajtu. Materiał genetyczny każdej odmiany jest charakteryzowany za pomocą specjalnie zdefiniowanego ciągu
oznaczeń substancji barwiących, które mają największy wpływ na barwę kwiatu tejże odmiany. Np. tulibajt biały ma
\textbf{charakterystykę} \underline{D}DAF\underline{6A}B3\underline{4A}D\underline{F},
natomiast tulibajt purpurowy B\underline{D}1\underline{6A4}36B\underline{AF}. Genetycy zauważyli, że najbardziej
intensywne barwy kwiatów uzyskuje się krzyżując takie odmiany, dla których subkod genetyczny jest opisany za pomocą co
najmniej 6 substancji (być może tych samych ale zapisanych na różnych pozycjach). Subkod genetyczny wyznaczony dla
dwóch odmian stanowi uporządkowany (najdłuższy z~możliwych) ciąg znaków, które w~niezmienionej kolejności występują
zarówno w~charakterystyce jednej jak i~drugiej odmiany. Np. subkodem tulibajtu białego i~purpurowego jest \underline{D6A4AF}, ale
również DA6BAF jak i~D6A3AF. Dwie odmiany mogą mieć kilka subkodów, ale każdy z~nich ma tyle samo znaków. Jeśli
subkod dwóch odmian ma co najmniej 6 znaków, skrzyżowanie pary da intensywną barwę kwiatu. Botanicy wyznaczyli
charakterystyki wielu odmian tulibajtu i~chcą sprawdzić, które z~nich warto krzyżować. Zwrócili się z~zadaniem do
informatyków.

\subsection*{Prosty przykład}

Po skrzyżowaniu dwóch odmian o charakterystykach odpowiednio
\underline{\textbf{D}}DAF\underline{\textbf{6A}}B3\underline{\textbf{4A}}D\underline{\textbf{F}} i B\underline{\textbf{D}}1\underline{\textbf{6A4}}36B\underline{\textbf{AF}}
uzyskamy odmianę z subkodem genetycznym
\underline{\textbf{D6A4AF}} albo D6A3AF albo D6ABAF albo DA6BAF.

\section*{Wejście}

W~pierwszym wierszu wejścia znajduje się całkowita liczba $n$ ($2 \leqslant n \leqslant 50$) oznaczająca ilość par odmian tulibajtu, dla których należy
policzyć długość subkodu.

W~kolejnych $2n$ liniach umieszczone są charakterystyki odmian tulibajtu. Dla każdej kolejnej pary charakterystyk należy
wyznaczyć długość subkodu.

\section*{Wyjście}

W~pierwszej linii wyjścia wypisz $n$ liczb (oddzielonych pojedynczą spacją) reprezentujących długości subkodów
wyznaczonych dla $n$ kolejnych par charakterystyk.

\section*{Przykład}
\begin{multicols*}{2}
dane wejściowe:
\begin{Verbatim}[commandchars=+\[\]]
3
+underline[45]A+underline[64]D+underline[5]F5+underline[B34]2
+underline[456]3+underline[45]AC6A+underline[B]7+underline[34]5
8+underline[7]9+underline[A]7D680A+underline[968]
6+underline[7A896]BD7+underline[8]9
+underline[5]67+underline[45]D6E+underline[7]4F+underline[56]
3+underline[5]2D+underline[4]E+underline[5]CA+underline[7]3B+underline[56]
\end{Verbatim}
wynik:
\begin{Verbatim}[commandchars=+\[\]]
8 6 6
\end{Verbatim}
\end{multicols*}
\end{document}
