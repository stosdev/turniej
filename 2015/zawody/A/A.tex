\documentclass{article}

\usepackage{amssymb}
\usepackage{polski}
\usepackage[polish]{babel}
\usepackage[utf8]{inputenc}
\usepackage[T1]{fontenc}
\usepackage{libertine}
\usepackage{graphicx}
\usepackage{multicol}
\usepackage{fancyhdr}
\usepackage[a4paper,headheight=65pt,bottom=65pt]{geometry}
\usepackage{lastpage}

\pagestyle{fancy}
\fancyhf{}
\fancyhead[L]{Podlaski Turniej w~Programowaniu Zespołowym 2015\\ Politechnika Białostocka, 21-22 marca 2015}
\fancyhead[R]{\includegraphics[scale=0.7]{logoturnieju.png}}
\fancyfoot[L]{Problem A: Bezpieczne liczby}
\fancyfoot[R]{\thepage/\pageref{LastPage}}
\renewcommand{\footrulewidth}{0.4pt}

\begin{document}
\begin{center}
  \begin{Huge}
    Problem A: Bezpieczne liczby
  \end{Huge}
\end{center}

Mały Jaś bardzo lubi matematykę, a~szczególnie ,,kręcą'' go liczby pierwsze. Przypomnijmy, że to
liczby naturalne mające dokładnie dwa różne dzielniki: jedynkę i~samą siebie. Już jako małe
dziecko na kartce papieru wyznaczał kolejne liczby pierwsze. Później czytał on o~fascynującej
historii odkrywania coraz większych liczb pierwszych. Jaś wie, że początkowo występują one
bardzo często, ale z~czasem odstępy między nimi rosną (aczkolwiek ten wzrost jest powolny).
Ostatnio tata powiedział Jasiowi o~tzw. bezpiecznych liczbach pierwszych, czyli takich $p$, że
$\lfloor \frac{p}{2}\rfloor$ jest również liczbą pierwszą. Tata wyjaśnił, że mają one szczególne zastosowanie w
kryptografii, co szczególnie zainteresowało Jasia, który w~przyszłości chce zostać matematykiem
lub informatykiem. Jaś stwierdził, że bezpiecznych liczb pierwszych jest dużo mniej od zwykłych
liczb pierwszych i~ich wyznaczanie przychodzi mu z~pewnym trudem. Interesują go zwłaszcza
liczności tych liczb w~poszczególnych przedziałach. Niestety nie jest on w~stanie poradzić sobie z
tym samodzielnie (przynajmniej dopóki nie nauczy się programować). Poprosił o~pomoc swego
taty, który jednak ostatnio jest bardzo zapracowany (nielegalnie manipuluje cyferkami na serwerach
bankowych). W~związku z~tym zwraca się on do swego kuzyna, czyli Ciebie o~pomoc w~tej
niecierpiącej zwłoki sprawie.

\section*{Wejście}

W~pierwszej linijce wejścia są podana jest jedna liczba całkowita $n$ ($1 \leqslant n \leqslant 10$) oznaczająca liczbę
zapytań Jasia. W~kolejnych $n$ liniach podane są po dwie liczby całkowite $a$~i~$b$ ($1 \leqslant a \leqslant b \leqslant 10^{12} , b-a \leqslant 10^7$)
oznaczające dolny i~górny kraniec interesującego Jasia przedziału.

\section*{Wyjście}
W~kolejnych $n$ liniach wyjścia należy wypisać po jednej liczbie całkowitej, która oznacza liczbę bezpiecznych liczb pierwszych należących do odpowiedniego przedziału <$a$, $b$>.

\section*{Przykład}
dane wejściowe:
\begin{verbatim}
4
1 10
11 20
21 30
1 100
\end{verbatim}
wynik:
\begin{verbatim}
2
1
1
7
\end{verbatim}
\end{document}
