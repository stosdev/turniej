\documentclass{article}

\usepackage{amssymb}
\usepackage{polski}
\usepackage[polish]{babel}
\usepackage[utf8]{inputenc}
\usepackage[T1]{fontenc}
\usepackage{libertine}
\usepackage{graphicx}
\usepackage{multicol}
\usepackage{fancyhdr}
\usepackage[a4paper,headheight=65pt,bottom=65pt]{geometry}
\usepackage{lastpage}

\pagestyle{fancy}
\fancyhf{}
\fancyhead[L]{Podlaski Turniej w~Programowaniu Zespołowym 2015\\ Politechnika Białostocka, 21-22 marca 2015}
\fancyhead[R]{\includegraphics[scale=0.7]{logoturnieju.png}}
\fancyfoot[L]{Problem C: Chodnik}
\fancyfoot[R]{\thepage/\pageref{LastPage}}
\renewcommand{\footrulewidth}{0.4pt}

\begin{document}
\begin{center}
  \begin{Huge}
    Problem C: Chodnik
  \end{Huge}
\end{center}

Robotnicy pracują przy układaniu chodnika. Na dworze upał, praca jest ciężka i~monotonna, a~szef
ciągle pogania. Dysponują oni prostokątnymi płytami chodnikowymi o~rozmiarze $2$ na $1$, a~mają za
zadanie ułożyć chodnik o~szerokości równej $4$. Jeden z~robotników (o~imieniu Zdzisław) zauważył,
że może układać płyty na różne sposoby i~tworzyć rozmaite mozaiki. Dzięki tym rozmyślaniom
praca stała się mniej nużąca. Spytał on nawet szefa, w~jaki sposób powinien układać płyty na co
tamten (najwyraźniej pozbawiony wyobraźni) odparł, że on widzi tylko jeden sposób ich ułożenia.
Gdy Zdzisław rzekł, że tych sposobów może być więcej niż ludzi na ziemi, to szef i~jego koledzy
wyśmiali biedaka, każąc mu się udać do psychiatry. A~może to właśnie oni powinni się tam udać?
Pomóż Zdzisławowi (który przecież nie jest programistą) i~napisz program obliczający ile jest
możliwości ułożenia chodnika o~zadanej długości.

\section*{Wejście}
W~pierwszej linii wejścia podana jest liczba $N$ ($1\leqslant N\leqslant 1000$) zapytań. W~każdej z~kolejnych $N$ linii
podana będzie jedna liczba całkowita $n$ ($1\leqslant n \leqslant 10^6$) oznaczająca długość układanego chodnika.

\section*{Wyjście}
W~$N$ liniach wyjścia powinny się pojawić odpowiedzi na kolejne zapytania. W~każdej linii jest to
liczba sposobów ułożenia chodnika modulo $10^{18}$.

\section*{Przykład}
dane wejściowe:
\begin{verbatim}
2
2
3
\end{verbatim}
wynik:
\begin{verbatim}
5
11
\end{verbatim}
\end{document}
